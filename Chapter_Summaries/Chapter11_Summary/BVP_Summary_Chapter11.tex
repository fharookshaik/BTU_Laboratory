\documentclass{article}

\usepackage[utf8]{inputenc}
\usepackage{geometry}
\usepackage{graphicx}
\usepackage{titling}
\usepackage{fancyhdr}
\usepackage{cmbright}
\usepackage{caption}
\usepackage{subcaption}

\geometry{
	a4paper,
	total={170mm, 257mm},
	left=20mm,
	top=20mm
}


\title{Chapter 11 : Rules and Regularities}
\author{Fharook Shaik}
\date{28 January 2025}

\fancypagestyle{fancy}{
	\fancyhf{}
	\fancyfoot[R]{\includegraphics[width=3cm]{images/BTULogo_englisch_grau_2x.png}}
	\fancyfoot[L]{\thedate}
	\fancyhead[L]{13869 - Braitenberg Vehicle Praktium}
	\fancyhead[R]{\theauthor}
}

\pagestyle{fancy}

\makeatletter
\renewcommand{\maketitle}{
	\thispagestyle{fancy}
	\null
	\vskip 1em
	\begin{center}
		{\LARGE \@title \par}
	\end{center}
	\vskip 3em
}
\makeatother


\begin{document}

	\maketitle

	\noindent\begin{tabular}{@{}ll}
		Student & \theauthor\\
		Professor &  Dr. Cunningham, Douglas\\
		Matrikel-Nr.: & 5014962
		 
	\end{tabular}

	\section*{Summary}

	Chapter 11 builds on the idea of "getting ideas" from the previous chapter and takes another step toward creating behavior that resembles thinking. While Vehicle 10 formed abstract concepts by recognizing patterns in its environment, Vehicle 11 goes further by understanding sequences of events and their regularities. This new ability allows the vehicle to predict outcomes, recognize causal relationships, and respond to its surroundings in a way that appears even more intelligent.

	The chapter begins by acknowledging skepticism many may still doubt that the vehicles are truly "thinking". While previous vehicles could generate complex behaviors and even uncover hidden patterns, they have yet to demonstrate anything resembling genuine reasoning. However, the author argues that this is just a step in a longer process. By gradually adding new mechanisms, vehicles will become increasingly capable of behaviors that approach real thought.

	\subsection*{Learning Sequences}

	Vehicle 11 introduces a significant improvement over its predecessors by incorporating a new type of connection in its brain. Previous vehicles relied on Mnemotrix, a special wire that strengthened connections between elements activated together, allowing vehicles to form associations between things that often co-occur. However, this system could not capture events that happen in sequence, such as lightning followed by thunder, or a vehicle encountering food and then consuming it. To address this limitation, Vehicle 11 is equipped with a new type of wire called Ergotrix, which conducts signals in only one direction and strengthens when one event regularly follows another.

	By using Ergotrix, Vehicle 11 can encode knowledge about sequences of events. For example, if the vehicle repeatedly encounters a light before hearing a sound, it will strengthen the connection between these two perceptions. Over time, the vehicle will begin to anticipate the second event whenever it detects the first. This system allows the vehicle to recognize cause-and-effect relationships, even if it does not fully understand their underlying mechanisms.

	This distinction between simultaneous associations (Mnemotrix) and sequential learning (Ergotrix) reflects an important difference in how knowledge is structured. The Mnemotrix system builds an understanding of "what things exist" grouping properties into concepts while the Ergotrix system captures "how things change" establishing rules about sequences and interactions. The two systems mirror the distinction between geography (which maps what is present) and history (which records how things develop over time).

	While it may seem that concepts must be learned first before rules can be discovered, the chapter highlights how the two processes influence each other. In human learning, for example, we must first recognize words before we can understand grammar, but at the same time, grammatical structure helps define what counts as a meaningful word. Similarly, Vehicle 11 benefits from an interaction between Mnemotrix and Ergotrix.

	The chapter suggests a method for integrating the two learning processes. Whenever an Ergotrix connection strengthens due to repeated sequences, the Mnemotrix connections within the involved concepts are also reinforced. This means that concepts become particularly well-defined when they appear in meaningful sequences. From an observer's perspective, this results in the vehicle responding more strongly to events that are known to have consequences. For instance, a vehicle approaching an obstacle at high speed may learn to react swiftly, having recognized the pattern of impact following rapid approach.

	Moreover, Vehicle 11 can learn social signals. It will quickly remember which of its own actions consistently provoke reactions from other vehicles, either encouraging or discouraging them. After a period of learning, the vehicle will start using these behaviors deliberately, either to communicate or to avoid unwanted consequences. Similarly, it will recognize premonitory signals events that consistently precede certain behaviors in others and react to them as if anticipating what is about to happen.

	\subsection*{Conclusion}

	Chapter 11 takes a major step toward intelligent behavior by introducing the ability to learn sequences and causal relationships. While previous vehicles could associate simultaneous events, Vehicle 11 can now recognize patterns in time, making it capable of prediction and adaptation. By distinguishing between concepts (Mnemotrix) and sequences (Ergotrix), the vehicle develops a structured understanding of its world, much like how humans learn both vocabulary and grammar.

	This interplay between abstract concepts and sequential rules allows the vehicle to refine its responses, anticipate future events, and engage in more sophisticated interactions with other vehicles. The chapter suggests that as these mechanisms continue to evolve, the vehicles will exhibit behaviors that are increasingly difficult to distinguish from real thought. With this foundation in place, the next steps will introduce even more elements of intelligence, bringing synthetic psychology closer to true cognition.
		
\end{document}